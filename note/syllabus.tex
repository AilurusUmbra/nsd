\documentclass[11pt,a4paper]{article}
\usepackage[inner=1.5cm,outer=1.5cm,top=1.5cm,bottom=1.5cm]{geometry}
\pagestyle{empty}
\usepackage{graphicx}
\usepackage{fancyhdr, lastpage, bbding, pmboxdraw}
\usepackage[printwatermark]{xwatermark}
\usepackage{xcolor}
%\usepackage[usenames,dvipsnames]{color}
%\definecolor{darkblue}{rgb}{0,0,.6}
%\definecolor{darkred}{rgb}{.7,0,0}
%\usepackage[colorlinks,pagebackref,pdfusetitle,urlcolor=darkblue,%
%citecolor=darkblue,linkcolor=darkred,bookmarksnumbered,plainpages=false]%
%{hyperref}
\usepackage[colorlinks,pagebackref,pdfusetitle,%
bookmarksnumbered,plainpages=false]%
{hyperref}
\renewcommand{\thefootnote}{\fnsymbol{footnote}}
\usepackage{acronym}

\usepackage{titlesec}
\titlespacing{\paragraph}{0pt}{.3cm}{1em}
\titlespacing{\subparagraph}{0pt}{.2cm}{1em}

\usepackage{enumitem}
\setlist{nosep} % or \setlist{noitemsep} to leave space around whole list

\newwatermark[allpages,color=black!15,angle=55,scale=5,xpos=0,ypos=0]%
{DRAFT}

\pagestyle{plain}

\begin{document}

\begin{center}
{\Large \textsc{Numerical Software Development}}
\end{center}
\begin{center}
2019
\end{center}
%\date{September 26, 2014}

\begin{center}
\rule{6in}{0.4pt}
\begin{minipage}[t]{.75\textwidth}
\begin{tabular}{llcccll}
\textbf{Instructor:} & Yung-Yu Chen & & &  &
\textbf{Time:} & TBD \\
\textbf{Email:} & \href{mailto:yyc@solvcon.net}{yyc@solvcon.net} & & & &
\textbf{Place:} & TBD
\end{tabular}
\end{minipage}
\rule{6in}{0.4pt}
\end{center}

%\noindent\textbf{Course Pages:} \url{http://yourWebPage1.com/teaching}

%\vskip.3cm
%\noindent\textbf{Office Hours:}

\paragraph{Objectives:}

This course discusses the art of engineering numerical software, which is
computer programs applying numerical methods for solving mathematical or
physical problems.  We will be using the combination of bash, git, Python, C++
(and a little bit of C), and related tools to learn the modern development
processes.

\paragraph{Prerequisites:}

This is a graduate or senior level course open to all students who have taken
engineering mathematics or equivalence.  Working knowledge of Linux and
Unix-like is required.  Prior knowledge to numerical methods is recommended.

\paragraph{Lectures:}
\begin{enumerate}
\item Python and numpy
\item C++ and computer architecture
\item Fundamental engineering tools
\item Memory management
\item Matrix operations
\item Cache optimization and SIMD
\item Recap: modern C++
\item Xtensor and numpy
\item Pybind11 and cpython API
\item Cpython internals
\item High-performance software design
\item Hybrid code organization
\end{enumerate}

\paragraph{Grading:} Homework 30\%, midterm exam: 30\%, project: 40\%.

\paragraph{References:}
\begin{itemize}
\item Numerical analysis
\item Effective Modern C++
\item Source code: xtensor, pybind11, numpy, cpython
\end{itemize}

%\paragraph{Important Dates:}

%\paragraph{Course Policy:}

%\paragraph{Class Policy:}

%\paragraph{Academic Honesty:}

\clearpage

\paragraph{Lecture Outlines:}

\subparagraph{Lecture 1: Python and numpy.}  Numerical software is always
developed as a platform.  It works like a library providing data structures and
helpers to solve problems.  The users will use a scripting engine it provides
to build applications.  Python is a popular choice for the scripting engine.

\begin{itemize}
\item Organize Python modules
\item Numpy for array-centric code
\item Singular value problems
\item The Laplace equations
\end{itemize}

\subparagraph{Lecture 2: C++ and computer architecture.}  The low-level code of
numerical software must be high-performance.  The industries chose C++ because
it can take advantage of everything that a hardware architecture offers while
using any level of abstraction.

\begin{itemize}
\item Introduce compiler tools
\item Real numbers
\item Von Neumann architecture
\item Memory hierarchy
\end{itemize}

\subparagraph{Lecture 3: fundamental engineering practices.}  Writing computer
code is only a fraction of the work to build computer software.  A large chunk
of resources is spent in making sure the software delivers the results it
should.  As in any engineering discipline, automation is key.

\begin{itemize}
\item Build system: make and cmake
\item Version control: git and concept of centralized VCS
\item Code review
\item Automatic testing: google-test, py.test, and TravisCI
\item Profiling: perf and timing
\end{itemize}

\subparagraph{Lecture 4: memory management.}  Numerical software tends to use
as much memory as a workstation has.  There are two major reasons: (i) the
applications usually need a lot of memory, and (ii) we trade memory for speed.

\begin{itemize}
\item Stack and heap
\item Linux and POSIX memory manager
\item C++ memory manager
\item Linux memory report
\end{itemize}

\subparagraph{Lecture 5: matrix operations.}  As linear algebra is fundamental
in almost everything uses mathematics, matrices are everywhere in numerical
analysis.  There isn't shortage of linear algebraic software packages and it's
critically important to understand how they work.

\begin{itemize}
\item Multi-dimensional array in C++ and other languages
\item Matrix-matrix and matrix-vector operations
\end{itemize}

\subparagraph{Lecture 6: cache optimization and SIMD.}  To use all the cycles
the data transfer between main memory and CPU must be minimized.

\begin{itemize}
\item Stride analysis; how cache greatly affects performance
\item Tiling
\item Single instruction multiple data (SIMD)
\end{itemize}

\subparagraph{Lecture 7: recap: modern C++.}  Some skills are important to
writing maintainable code in C++.  We should stick to modern C++, that is the
dialect exclusively uses C++ standard 11, 14, 17, and beyond.

\begin{itemize}
\item Curiously recursive template pattern (CRTP): static polymorphism
\item Smart pointers
\item Template meta-programming and compile-time computation
\item Perfect forwarding
\end{itemize}

\subparagraph{Lecture 8: xtensor and numpy.}  We need the compile-time
optimization enabled by static typing information, which is something numpy and
Python cannot provide.  Xtensor, instead, gives it to us.

\begin{itemize}
\item C++ array operations
\item Dynamic and static arrays
\item Ownership management across C++ and Python
\end{itemize}

\subparagraph{Lecture 9: pybind11 and cpython API.}  High-performance code
needs careful design for the data flow between the low-level implementation and
the high-level scripting layer.

\begin{itemize}
\item Reference counting
\item Introduce pybind11
\item Cpython and pybind11 API for important Python types
\end{itemize}

\subparagraph{Lecture 10: cpython internals.}  Learn when Python is slow and
when it can be worked around.

\begin{itemize}
\item Primitive numbers
\item Tuple, list, and dict
\end{itemize}

\subparagraph{Lecture 11: high-performance software design.}  There are some
general guidelines for high-performance code.  Although following them doesn't
prevent you from slow code, ignoring them usually guarantees it.

\begin{itemize}
\item Array majoring
\item Compact struct and encapsulating it
\item Profiling and diagnostic hook
\end{itemize}

\subparagraph{Lecture 12: hybrid code organization.}  Numerical software is
born to be fast.  In the development iterations performance always involves.
This is the central idea of developing a hybrid system, and the code should be
organized accordingly.

\begin{itemize}
\item Wrap around pybind11
\item Straight dependency
\end{itemize}

\end{document}
